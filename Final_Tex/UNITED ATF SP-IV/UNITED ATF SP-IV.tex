\documentclass[12pt,a4paper]{article}
\usepackage[english,russian]{babel}
\usepackage[utf8]{inputenc}
\usepackage[T1]{fontenc}
\usepackage[top=2cm, bottom=2cm, left=2cm, right=1.5cm]{geometry}
\usepackage{array}
\usepackage{tabularx}
\usepackage{tipa}
\renewcommand{\baselinestretch}{1}
\setlength{\parindent}{0em}
\setlength{\parskip}{0.5em}
\usepackage{setspace}
\usepackage{enumitem}
\usepackage{textcomp}
\usepackage{stackengine}

\begin{document}
\begin{spacing}{2.5}
{\textbf{\fontsize{20pt}{30pt}\selectfont United ATF SP-IV}}
\end{spacing}

\begin{spacing}{2.5}
{\textbf{\fontsize{14pt}{4pt}\selectfont Описание продукта:}}
\end{spacing}

{\textbf{United ATF SP-IV - это синтетическая жидкость для автоматических трансмиссий, специально разработанная с использованием высокоэффективных синтетических базовых масел и продвинутой технологии производства присадок в целях обеспечения оптимальной производительности и защиты автоматических трансмиссий. Она соответствует всем требованиям спецификации ATF SP-IV, которой удовлетворяют большинство трансмиссионных масел для автоматических коробок передач, используемых в автомобилях Hyundai, Kia и Mitsubishi.}}

Оптимизированные фрикционные свойства United ATF SP-IV обеспечивают плавное и комфортное переключение передач путём снижения вибрации. Превосходные термическая стабильность и устойчивость к окислению поддерживают чистоту трансмиссии, продлевая ресурс жидкости и обеспечивая высокую производительность даже в трудных условиях движения. Сверхтекучесть при низких температурах помогает снизить потребление топлива и добиться мгновенной защиты при холодном пуске.

\bigskip
{\textbf{\fontsize{14pt}{10pt}\selectfont Преимущества:}}

\begin{itemize}[itemsep=1.4pt]

\item  Превосходная термическая стабильность и устойчивость к окислению.
\item  Продвинутая защита от износа, обеспечивающая защиту важных узлов трансмиссий.
\item  Защита от коррозии и пенообразования.
\item  Сверхтекучесть при низких температурах
\item  Улучшенное снижение трения.
 
\end{itemize}

\begin{spacing}{1}
{\textbf{\fontsize{14pt}{4pt}\selectfont Характеристики:}}
\end{spacing}

\renewcommand*{\arraystretch}{1.1}
\addstackgap[15pt]{\begin{tabularx}{\textwidth}{m{6.9cm} X m{2cm}}
\textbf{Описание теста} & \textbf{Метод} & \\
Удельная масса при 15\textdegree{}C & ASTM D 4052 & 0.845\\
Точка воспламенения, \textdegree{}C & ASTM D & 198\\
Точка изменения текучести, \textdegree{}C & ASTM D 97 & -45\\
Кинематическая вязкость при 40\textdegree{}C & ASTM D 445 & 26.72\\
\multicolumn{1}{r}{при 100\textdegree{}C} & ASTM D 445 & 5.66\\
Индекс вязкости & ASTM D 2270 & 160 \\
Цвет & ASTM D 1500 & <3.5\\
\end{tabularx}}

\begin{spacing}{0.5}
{\textbf{\fontsize{14pt}{4pt}\selectfont Допуски, одобрения, рекомендации:
}}
\end{spacing}
\begin{itemize}[itemsep=1pt]
\item Класс ATP SP-IV 
\end{itemize}

\end{document}
