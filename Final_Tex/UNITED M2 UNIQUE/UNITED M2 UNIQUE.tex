\documentclass[12pt,a4paper]{article}
\usepackage[english,russian]{babel}
\usepackage[utf8]{inputenc}
\usepackage[T1]{fontenc}
\usepackage[top=2cm, bottom=2cm, left=2cm, right=1.5cm]{geometry}
\usepackage{array}
\usepackage{tabularx}
\usepackage{tipa}
\renewcommand{\baselinestretch}{1}
\setlength{\parindent}{0em}
\setlength{\parskip}{0.5em}
\usepackage{setspace}
\usepackage{enumitem}
\usepackage{textcomp}
\usepackage{stackengine}

\begin{document}
\begin{spacing}{2.5}
{\textbf{\fontsize{20pt}{30pt}\selectfont UNITED M2 UNIQUE}}
\end{spacing}

\begin{spacing}{2.5}
{\textbf{\fontsize{14pt}{4pt}\selectfont Описание продукта:}}
\end{spacing}

{\textbf{United M2 Unique — это полностью синтетическое премиумное моторное масло, обладающее топливосберегающими свойствами и специально разработанное для использования в бензиновых или дизельных двигателях, и особенности в двигателях Volkswagen TDI с системой прямого впрыска насос-форсунками.}

Полностью синтетическое моторное масло United M2 Unique идеально подходит для современных высокопроизводительных малотоксичных двигателей с двойным верхним распределительным валом, оснащенных турбонаддувом\textbackslash  турбонагнетателем, и двигателей с системой прямого впрыска. Данные двигатели широко распространены на современных европейских и японских автомобилях. 

\bigskip
{\textbf{\fontsize{14pt}{10pt}\selectfont Преимущества:}}


\begin{itemize}[itemsep=1.4pt]
\item Высокий индекс вязкоcти и низкая летучесть обеспечивают сверхтекучесть в широком диапазоне температур и условий эксплуатации.
\item Улучшенная моющая способность и повышенные противоизносные качества увеличивают производительность двигателя без дополнительных энергозатрат.
\item Улучшенное снижение трения, особенно в момент холодного пуска, и обеспечение качественной смазки всех частей двигателя.
\item Превосходные противоокислительные свойства предотвращают возникновение коррозии и ржавчины, увеличивая интервал между заменами масла. 
\end{itemize}

\begin{spacing}{1}
{\textbf{\fontsize{14pt}{4pt}\selectfont Характеристики:}}
\end{spacing}

\renewcommand*{\arraystretch}{1.1}
\addstackgap[15pt]{\begin{tabularx}{\textwidth}{m{6.92cm} m{5cm} m{2cm} m{2cm}}
\textbf{Описание теста} & \textbf{Метод} & \\
SAE вязкость масла & SAE J 300 & \textbf{5W30} & \textbf{5W40}\\
Удельная масса при 15\textdegree{}C & ASTM D 4052 & 0.858 & 0.861\\
Точка воспламенения, \textdegree{}C & ASTM D 92 & 212 & 220\\
Точка изменения текучести, \textdegree{}C & ASTM D 97 & -39 & -39\\
Кинематическая вязкость при 40\textdegree{}C & ASTM D 445 & 70.1 & 88.9\\
\multicolumn{1}{r}{при 100\textdegree{}C} & ASTM D 445 & 11.8 & 14.5\\
Индекс вязкости & ASTM D 2270 & 165 & 170 \\
Общее щелочное число (mgKOH/g) & ASTM D 2896 & 10.8 & 10.8\\
Цвет & ASTM D 1500 & <2.5 & <2.5\\
\end{tabularx}}

\begin{spacing}{8}
\end{spacing}

\begin{spacing}{0.5}
{\textbf{\fontsize{14pt}{4pt}\selectfont Допуски, одобрения, рекомендации:
}}
\end{spacing}
\begin{itemize}[itemsep=1pt]
\item API SN/CF
\item ACEA A3/B4-12
\item ACEA A3/B3-12
\item Лист допуска 229.3
\item VW 502 00/505 00
\item Renault 0700
\item Renault 0710
\end{itemize}

\end{document}
