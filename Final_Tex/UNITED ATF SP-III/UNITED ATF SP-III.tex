\documentclass[12pt,a4paper]{article}
\usepackage[english,russian]{babel}
\usepackage[utf8]{inputenc}
\usepackage[T1]{fontenc}
\usepackage[top=2cm, bottom=2cm, left=2cm, right=1.5cm]{geometry}
\usepackage{array}
\usepackage{tabularx}
\usepackage{tipa}
\renewcommand{\baselinestretch}{1}
\setlength{\parindent}{0em}
\setlength{\parskip}{0.5em}
\usepackage{setspace}
\usepackage{enumitem}
\usepackage{textcomp}
\usepackage{stackengine}

\begin{document}
\begin{spacing}{2.5}
{\textbf{\fontsize{20pt}{30pt}\selectfont UNITED ATF SP-III}}
\end{spacing}

\begin{spacing}{2.5}
{\textbf{\fontsize{14pt}{4pt}\selectfont Описание продукта:}}
\end{spacing}

{\textbf{United ATF SP-III - жидкость для автоматических трансмиссий, специально разработанная в целях соответствия требованиям спецификации SPIII, которым отвечают жидкости, используемые в автомобилях с автоматической трансмиссией Hyundai, Kia и Mitsubishi.  United ATF SP-III изготавливается из синтетического базового сырья с высоким индексом вязкости, а также присадок, произведенных по новейшей технологии, что обеспечивает наилучшие фрикционные характеристики в момент переключения передач, превосходные свойства при низких температурах, противоокислительную и химическую стабильности в течение всего эксплуатационного периода. }

United ATF SP-III рекомендуется для использования в автоматических трансмиссиях, системах рулевого управления с усилителем, а также гидравлических приводах легковых, внедорожных и коммерческих автомобилей Mutsubishi, Hyunda и Kia, требующих жидкости типа SPIII.

\bigskip
{\textbf{\fontsize{14pt}{10pt}\selectfont Преимущества:}}

\begin{itemize}[itemsep=1.4pt]

\item  Превосходная термическая стабильность и устойчивость к окислению.
\item  Продвинутая защита от износа, обеспечивающая защиту важных узлов трансмиссий.
\item  Защита от коррозии и пенообразования.
\item  Сверхтекучесть при низких температурах
\item  Улучшенное снижение трения.
 
\end{itemize}

\begin{spacing}{1}
{\textbf{\fontsize{14pt}{4pt}\selectfont Характеристики:}}
\end{spacing}

\renewcommand*{\arraystretch}{1.1}
\addstackgap[15pt]{\begin{tabularx}{\textwidth}{m{6.9cm} X m{2cm}}
\textbf{Описание теста} & \textbf{Метод} & \\
Удельная масса при 15\textdegree{}C & ASTM D 4052 & 0.853\\
Точка воспламенения, \textdegree{}C & ASTM D & 198\\
Точка изменения текучести, \textdegree{}C & ASTM D 97 & -45\\
Кинематическая вязкость при 40\textdegree{}C & ASTM D 445 & 38.1\\
\multicolumn{1}{r}{при 100\textdegree{}C} & ASTM D 445 & 7.24\\
Индекс вязкости & ASTM D 2270 & 157 \\
Цвет & ASTM D 1500 & Red\\
\end{tabularx}}

\begin{spacing}{0.5}
{\textbf{\fontsize{14pt}{4pt}\selectfont Допуски, одобрения, рекомендации:
}}
\end{spacing}
\begin{itemize}[itemsep=1pt]
\item Класс ATP SP-III 
\end{itemize}

\end{document}
