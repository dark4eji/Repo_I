\documentclass{article}
\linespread{1.25}
\begin{document}

\tableofcontents

\title{Document for practice}
\author{A. V. Savin}
\date{}
\maketitle

\begin{abstract}
Thou shalt not make thee any graven image, or any likeness of any thing that is in heaven above, or that is in the earth beneath, or that is in the waters beneath the earth
\end{abstract}

\section{Python}

{\setlength{\parindent}{1cm}}Python is a widely used high-level programming language for general-purpose programming, created by Guido van Rossum and first released in 1991. An interpreted language, Python has a design philosophy which emphasizes code readability (notably using whitespace indentation to delimit code blocks rather than curly brackets or keywords), and a syntax which allows programmers to express concepts in fewer lines of code than might be used in languages such as C++ or Java.[22][23] The language provides constructs intended to enable writing clear programs on both a small and large scale.[24]

\subsection[Some facts about Python]{Paradigm}{\setlength{\parindent}{1cm}}

Python features a dynamic type system and automatic memory management and supports multiple programming paradigms, including object-oriented, imperative, functional programming, and procedural styles. It has a large and comprehensive standard library.[25]
{\setlength{\parindent}{1cm}}
Python interpreters are available for many operating systems, allowing Python code to run on a wide variety of systems. CPython, the reference implementation of Python, is open source software[26] and has a community-based development model, as do nearly all of its variant implementations. CPython is managed by the non-profit Python Software Foundation.

\section{Django}{\setlength{\parindent}{1cm}}

{\setlength{\parindent}{1cm}}Django (/ˈdʒæŋɡoʊ/ JANG-goh)[5] is a free and open-source web framework, written in Python, which follows the model-view-template (MVT) architectural pattern.[6][7] It is maintained by the Django Software Foundation (DSF), an independent organization established as a 501(c)(3) non-profit.

\section[Columbia River]{The Columbia River}{\setlength{\parindent}{1cm}}

{\setlength{\parindent}{1cm}}The Columbia River is the largest river in the Pacific Northwest region of North America. Rising in the Rocky Mountains, it flows south into Washington, then turns west to form most of that state's border with Oregon before emptying into the Pacific, 1,243 miles (2,000 km) from its source. By volume it is the fourth-largest river in the US and the largest in North America that enters the Pacific. The river system hosts salmon and other fish that migrate between freshwater habitats and the saline waters of the Pacific Ocean. In the late 18th century, a private American ship became the first non-indigenous vessel to enter the river. Overland explorers entered the Willamette Valley through the scenic but treacherous Columbia River Gorge. Railroads were built in the valley in the late 19th century, many running along the river. Since the early 20th century, the river has been dammed for power generation, navigation, irrigation, and flood control. The 14 hydroelectric dams on the Columbia (Bonneville Dam pictured), the Snake River, and the Columbia's other tributaries produce more than 44 percent of total US hydroelectric power.

\end{document}