\documentclass[14pt,a4paper]{article}
\usepackage[english,russian]{babel}
\usepackage[utf8]{inputenc}
\usepackage[T1]{fontenc}
\usepackage[top=2cm, bottom=2cm, left=2cm, right=1.5cm]{geometry}
\usepackage{array}
\usepackage{tabularx}
\usepackage{tipa}
\newcommand{\tabitem}{~~\llap{\textbullet}~~}
\begin{document}

{
\linespread{3}
\renewcommand\arraystretch{1}

\fontsize{12pt}{5.5pt}\selectfont
\fontfamily{lmss}\selectfont

\begin{tabularx}{\textwidth}{| m{4cm} | m{3cm} | X |}
 \hline
extraneous & & \tabitem чуждый, посторонний\\
& & \tabitem стоящий вне (чего-л.); не связанный (с чем-л.)\\
& & \tabitem ненужный, лишний\\ \hline 
 intelligence~summary & & разведывательная сводка; разведсводка\\ \hline
 change into & & переходить\\ \hline
 specifier & & спецификатор\\ \hline
 excerpt & & отрывок\\ \hline
 guillments & ['gɪlɪmɔt] & «» фрацузские кавычки; ёлочки\\ \hline

\end{tabularx}}

\end{document}}

